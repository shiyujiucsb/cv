\documentclass{letter}

\usepackage{fullpage}

\begin{document}

{\bf Academic Personal Statement}\\
{\bf Shiyu Ji, applying Computer Science PhD}\\
\rule{\textwidth}{1pt}

Now I am a third year graduate student at Oklahoma State University (OSU), working on incentive issues, data security of crowdsourcing and wireless networks. Before that, I obtained my B.E. from Harbin Institute of Technology (HIT), China.

I am the first author of 5 papers. 4 of them have been published or accepted by prestigious conferences and journals like INFOCOM, ICC, IEEE Transactions on Mobile Computing (TMC) and Journal of Electronic Imaging (JEI).

I have quite a long history doing research in computer science. When I was a senior, I joined one of the CS labs in HIT as an undergraduate researcher. I proposed the idea that one may apply the compound chaotic cryptography to encrypt the bitmap images. As a result, I built the chaotic cryptosystem that can encrypt images of various formats, such as JPEG and GIF. I also tested my system based on NIST SP 800-22, and my cryptosystem satisfied all of the test criteria. I presented the results in the paper \emph{Image encryption schemes for joint photographic experts group and graphics interchange format formats based on three-dimensional baker with compound chaotic sequence generator}, which was published in JEI.

Then I joined OSU as a graduate student, and I continued my research with Dr. Chen as my advisor. Dr. Chen is an expert on economic incentives and cyber security in wireless networks and crowdsourcing applications. We did some research on the wormhole attacks in wireless network coding systems and incentive mechanisms in crowdsourcing/crowdsensing.

For the research of wormhole attacks, I researched the existing solutions to defend against wormhole attacks, and I found that the background scenarios of most the existing solutions were basically classical networks, which can be abstracted into graph models. However, for wireless network coding (WNC) systems, the classical graph models are very likely to fail when we analyze the traffic in WNC system. Therefore, many existing solutions that used the time or location information to detect wormhole attackers will probably not work in WNC settings. To address this problem, I proposed two algorithms, one was centralized and the other was distributed, to leverage the Expected Transmission Count (ETX) as a metric to detect wormhole attackers. Both the algorithms were heuristic. To verify their correctness, I rigorously proved the theoretical lower bound of successful rate for each of the algorithms. To verify the algorithms' performance, I also did a lot of simulations and the obtained results revealed the algorithms had sufficiently high successful rates. The results were presented in the following papers: \emph{DAWN: Defending Against Wormhole Attacks in Wireless Network Coding Systems}, which was presented by myself in INFOCOM 2014, and \emph{Wormhole Attack Detection Algorithms in Wireless Network Coding Systems}, which was accepted by TMC.

For the research of incentive mechanisms in crowdsourcing and crowdsensing, I mainly researched two problems:
\begin{itemize}
\item \emph{Sensing Uncertainty}: I researched the existing crowdsourcing incentive mechanisms, and I found that most of the classical mechanisms had few discussions on the sensing uncertainty, which is a prevailing phenomenon in crowdsourcing applications. Unfortunately, I found that the perturbations of sensing uncertainty have significant negative impacts on the crowdsourcing incentive mechanisms. To address this problem, I proposed a heuristic incentive mechanism which can achieve Trembling-Hand Perfect Equilibrium, even though there are significant uncertain perturbations in the crowdsourcing system. I also did a lot of simulations to verify the correctness and efficiency of the proposed incentive mechanism. The above results were presented in the paper \emph{Crowdsourcing with Trembles: Incentive Mechanisms for Mobile Phones with Uncertain Sensing Time}, which will be published in some conference or transaction in the future.
\item \emph{Finite Sensing Precisions}: This problem was about the finite sensing precisions, which could significantly reduce the performance of the classical crowdsensing incentive mechanisms. To handle with this issue, I proposed an incentive mechanism that can achieve Bayesian Nash Equilibrium, although each participatory user can only sense with limited precisions. I also did extensive simulations to verify the correctness and efficiency of the proposed incentive mechanism. The above results were presented in the paper \emph{Crowdsensing Incentive Mechanisms for Mobile Systems with Finite Precisions}, which was presented in ICC 2014.
\end{itemize}

Besides the academic research, I also have 2-year experience working as a TA in the Department of Computer Science, OSU. In my first TA year, I was committed the duty to grade for three theoretical courses \emph{Data Structures and Algorithms}, \emph{Formal Languages} and \emph{Numerical Analysis}. During this period, I learned a lot of theoretical knowledge in CS, and more importantly, the way to reason the theorems appropriately, which was quite helpful for my paper writing. In my second TA year, I was the TA for the two graduate level courses \emph{Introduction to Cryptography} and \emph{Wireless Networks}. I participated in preparing the course materials, including the introductory slides of ns3 and Crypto++. I was also responsible to give the students proper guide when they had some doubts of the assignments and term projects. In general, the TA experience let me know how to communicate with others in an academic atmosphere.

Now I am applying the graduate program in your university, basically because my current advisor has moved to another institute. I hope it will be a good change in my life. By attending the graduate program of your honorable university, I can continue my research with much more depth, and expand my research interests to gain the latest knowledge in the computer science world. Furthermore, it will definitely give me a lot of benefit if I could see and talk to the well-known specialists and scholars in your university. My ultimate goal is to become a professor in a university, and continue my research on computer science. Therefore, my PhD experience is required to have sufficient academic training. I think the prospective life as a graduate student in your university will be extraordinarily important to my future career.

Shiyu Ji, Graduate Student,\\
Department of Computer Science\\
Oklahoma State University\\
shiyu@cs.okstate.edu\\
\today{}
\end{document}