\documentclass{letter}

%\usepackage{setspace}
\usepackage{fullpage}

\begin{document}

%\setlength{\parindent}{0cm}
%\setlength{\parskip}{\baselineskip}
%\pagestyle{empty}

To Whom It May Concern,

Here I recommend Shiyu Ji to apply for the graduate program of your university. I have cooperated with him on the papers \emph{DAWN: Defending Against Wormhole Attacks in Wireless Network Coding Systems} and \emph{Wormhole Attack Detection Algorithms in Wireless Network Coding Systems}. Based on the cooperation experience, I feel Shiyu has great potential in research.

Both the papers, that we have cooperated on, research the threat model called wormhole attacks. In particular, we choose the wireless network coding as the scenario of the network setting. Most existing works of wormhole attacks focus on the classical networks, which can be represented in graph model. For example, the connectivity between the nodes is denoted by an edge. However, network coding is a totally different story, in which each pair of nodes have a link loss probability between them. Shiyu has verified that many classical solutions to defend against wormhole attacks will probably not work any more when it comes to network coding systems. Therefore, we tried to find some other advanced tools which can help us. After a lot of discussions, we decided to use the Expected Transmission Count (ETX) as a kind of metric to detect the existence of wormhole attackers. The idea was heuristic at first, but after Shiyu does some quantitative analysis, the results now become rigorous. For the two algorithms in the papers, he gives the theoretical lower bounds of the successful rates, which make the papers much more convincing. All the proofs are done by Shiyu. Some of his results are quite elegant. Therefore, I can say Shiyu has a good sense of reasoning and a strong capability of analysis.

Shiyu has also been responsible for all the simulations in both the papers. In order to verify the negative impacts of wormhole attacks, Shiyu has done extensive simulations, which successfully verify the significance of the negative influences caused by wormhole attacks. How to elegantly process the gained results is another story. The images in the first draft of the \emph{DAWN} paper were all unclear bitmaps, and they were not appropriate for publishing in proceedings or transactions. To solve this problem, within a very short period, Shiyu learned how to generate vector images by purely programming all by himself. In order to give a nice presentation of the simulation results, Shiyu has used extensive tools such as gnuplot, matplotlib, matlab, graphviz, etc. As a result, the simulation results are now presented in elegant images. The readers can see the details very clearly. I believe this gives the papers much bonus. Based on this fact, I can say Shiyu is very eager to learn something new, and he has strong experimental skills, especially programming.

Now Shiyu would like to apply the graduate programs since his current advisor is moving to another university. Generally it is a good idea to do the advanced research in a prestigious university. So I strongly encourage Shiyu to apply the graduate programs, especially PhD program. Since he has demonstrated remarkable academic skills in the past two years, I believe he can complete the degree in your university, and he will enjoy the study life in the future.

Sincerely,

% Sign here.
\vspace{1cm}

Professor, Ph.D. Advisor, Ph.D.\\
Department of Computer Science and Technology\\
Nanjing University\\
\today{}
\end{document}