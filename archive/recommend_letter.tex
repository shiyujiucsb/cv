\documentclass{article}

\usepackage{setspace}
\usepackage{fullpage}

\begin{document}

\setlength{\parindent}{0cm}
\setlength{\parskip}{\baselineskip}
\pagestyle{empty}

To Whom It May Concern,

I am writing to support Shiyu Ji's application for the graduate program of your university. Shiyu has been working with me in wireless networks and crowdsourcing since Fall 2012. As his advisor, I know his research qualifications very well.

Shiyu has demonstrated his remarkable research skills in the first two years' graduate coursework. His current GPA is 4.0/4.0, and he has passed the Comprehensive Exam and become a PhD candidate. He has finished 5 papers as the first author, and 4 of them have been published or accepted by prestigious conferences and journals like INFOCOM, ICC, TMC and Journal of Electronic Imaging. Shiyu owns distinguished ability to research new problems, and his achievements are outstanding in our department.

Three of his finished papers are dedicated to address the incentive issues in crowdsourcing. He leverages the game theoretical approaches to solve the problems of \emph{Sensing Uncertainty} and \emph{Finite Sensing Precisions}. During the research work, he has shown his great analytical and experimental skills.

For the Sensing Uncertainty project, he has studied the existing crowdsourcing game theoretical methods, which barely addressed the problems of sensing uncertainty. In order to address the negative influences which caused by uncertain perturbations, he has proposed an incentive mechanism that can achieve Trembling-Hand Perfect Equilibrium for one of the general game models of crowdsourcing. Shiyu has also been committed to the simulation works, and his results have successfully verified the correctness and efficiency of the algorithms in the paper.

For the Finite Sensing Precisions, he has studied another question, which is about finite precisions on crowdsensing. This question is prevailing in the real world but not yet addressed by most existing crowdsensing solutions. Similar to the Sensing Uncertainty project, Shiyu has proposed an incentive mechanism that can achieve Bayesian Nash Equilibrium. Again, he has also finished the simulations all by himself, and again his results have convinced the correctness and efficiency of the proposed algorithms.

Another two finished papers, which have been presented in INFOCOM 2014 and accepted by TMC respectively, are talking about the wormhole attack in wireless network coding systems. Currently there are many solutions to defend against wormhole attacks, but few of them are talking about network coding settings. To address this point, Shiyu has proposed the centralized and distributed Expected Transmission Count (ETX) based algorithms to defend against wormhole attacks in wireless network coding systems. To verify the theoretical results in the papers, Shiyu has done a lot of experiments and the gained results are convincing enough to persuade most of the readers.

Shiyu was also the TA for two of my graduate level courses, \emph{Introduction to Cryptography} and \emph{Wireless Networks}. He participated in grading the homeworks and exams and answering the students' questions. He also prepared some course materials. For example, in \emph{Introduction to Cryptography}, we needed to use Crypto++ to finish one programming assignments. But most of the students had no experience using crypto libraries before. To cope with this situation, Shiyu prepared the introductory slides of Crypto++ to help the students understand how to use the crypto library. To deal with the inconsistencies among various compilers, Shiyu recommended the students to use the High Performance Computing (HPC), which was offered by the university, and he figured out the configuration details of user interface on HPC all by himself. He shared his findings to the class, which definitely benefited the students a lot. In \emph{Wireless Networks}, the term project needed ns3, a network simulator. However again, most of the students had no background on this simulator. To help the students complete the project, Shiyu wrote the introductory slides of ns3, which successfully resolved most of the doubts that the students might had. Based on the experience that we worked together, I can say Shiyu is a nice TA, with adequate technical knowledge and patience.

I believe it will benefit his career a lot if he can communicate with active researchers directly in your university. I support him without reservation in this application. I will be glad to offer further information about him if you desire so.

%I am writing to apply for the US National Science Foundation (NSF) travel grant opportunity for US-based students (Type B) for INFOCOM 2014. I have the paper \emph{DAWN: Defending Against Wormhole Attacks in Wireless Network Coding Systems} accepted by INFOCOM 2014.

%I am a second year PhD candidate at Oklahoma State University, working on incentive issues, data privacy and cyber security of crowdsourcing and wireless networks.

%I have been working on the incentive mechanisms of crowdsourcing. I have investigated the problems on crowdsourcing such as \emph{Sensing Uncertainty} and \emph{Finite Sensing Precisions}. I have proposed different mechanisms that can achieve equilibria for various scenarios. I have also studied the privacy preserving mechanisms in crowdsourcing that can achieve k-anonymity and differential privacy. For the cyber security in wireless networks, I have researched the wormhole attack in network coding systems, and participated in designing the algorithms to defend against wormhole attacks.

%I have authored/coauthored 4 papers. 3 have been published or accepted by prestigious conferences and journals like INFOCOM, ICC and Journal of Electronic Imaging.

%By attending the prestigious conferences like INFOCOM, I can gain abundant advanced knowledge from the face-to-face discussions with the active contributors in my research field. Getting to know the people in the same field can definitely help me prepare the academic career in the future.

Sincerely,

%{\underline{\bf Please Sign Here!}}\\% Sign here.
\vspace{1cm}

Department of Computer Science\\
Oklahoma State University\\
\today{}
\end{document}