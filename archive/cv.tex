\documentclass{res}

\begin{document}

\pagestyle{empty}

%{\bf Curriculum Vitae}\\
%{\bf Shiyu Ji, applying Computer Science PhD}\\
%\rule{\textwidth}{1pt}

\name{SHIYU JI}
\address{Graduate Student\\Department of Computer Science\\Oklahoma State University}
\address{MSCS 314, Stillwater, OK 74075\\cs.okstate.edu/$\sim$shiyu\\shiyu@cs.okstate.edu}

\begin{resume}

\section{RESEARCH INTERESTS}
My current research interests include incentive mechanisms, data privacy and cyber security of crowdsourcing and wireless networks.

\section{EDUCATION}
\begin{itemize}
\item \textbf{Graduate Student} (2012-Present), Computer Science, Oklahoma State University.\\ 
Advisor: Dr. Tingting Chen.
\item \textbf{B.E.} (2008-2012), Information Security, Harbin Institute of Technology. GPA: 86.6/100.
\end{itemize}

\section{PUBLICATIONS}
\begin{enumerate}
\item Shiyu Ji, Tingting Chen, Sheng Zhong, Subhash Kak. DAWN: Defending Against Wormhole Attacks in Wireless Network Coding Systems, accepted by \emph{IEEE International Conference on Computer Communications (INFOCOM)}, Toronto, Canada, April, 2014.

\item Shiyu Ji, Tingting Chen. Crowdsensing Incentive Mechanisms for Mobile Systems with Finite Precisions, accepted by \emph{IEEE ICC 2014 - Mobile and Wireless Networking Symposium (ICC'14 MWN)}, Sydney, Australia, June, 2014.

\item Shiyu Ji, Tingting Chen, Sheng Zhong. Wormhole Attack Detection Algorithms in Wireless Network Coding Systems, accepted by \emph{IEEE Transactions on Mobile Computing (TMC)}, May, 2014.

\item Shiyu Ji, Xiaojun Tong, Miao Zhang. Image encryption schemes for joint photographic experts group and graphics interchange format formats based on three-dimensional baker with compound chaotic sequence generator. \emph{Journal of Electronic Imaging}, vol. 22, no. 1 (2013): 013017-013017.
\end{enumerate}
\section{EXPERIENCE}
\begin{itemize}
\item \textbf{Research Assistant} (06/2013-08/2013) at Oklahoma State University.\\
Advisor: Dr. Tingting Chen.\\
I worked on wormhole attacks and incentive issues of crowdsourcing.

For the research of wormhole attacks, I quantified the negative impacts of wormhole attacks on network coding systems and proposed the Expected Transmission Count (ETX) based algorithms to defend against wormhole attacks for network coding systems. Extensive simulations have verified the validness and efficiency of the proposed algorithms.

For incentive issues of crowdsourcing, I researched two problems:
\begin{itemize}
\item \emph{Sensing Uncertainty}: I investigated the negative impacts of uncertainty on crowdsourcing, and proposed the incentive mechanism that can achieve Trembling-Hand Perfect Equilibrium for the crowdsourcing game with uncertain perturbations. I also designed extensive simulations to successfully verify the correctness and efficiency of the proposed incentive mechanism.
\item \emph{Finite Sensing Precisions}: I investigated the harmful influences of finite precisions on crowdsensing, and proposed the incentive mechanism that can achieve Bayesian Nash Equilibrium for the crowdsensing game with finite sensing precisions. Extensive simulations have verified the proposed mechanism is valid and efficient.
\end{itemize}
\item \textbf{Teaching Assistant} (08/2012-Present) at Oklahoma State University.\\
I graded for \emph{Data Structures and Algorithms}, \emph{Formal Languages}, \emph{Numerical Analysis}, \emph{Introduction to Cryptography} and \emph{Wireless Networks}. I prepared the introductory slides of ns3 at the graduate level course of wireless network. I also guided the students to use Crypto++ in the intro to cryptography.
\item \textbf{Research Assistant} (11/2011-05/2012) at Harbin Institute of Technology.\\
Advisor: Prof. Xiaojun Tong.\\
I worked on the algorithms of chaotic image encryption on various graphic formats, such as JPEG, GIF, etc. I proposed an image encryption algorithm that leverages three-dimensional baker map and compound chaotic sequence generator. Extensive experiments have verified that the proposed encryption algorithm achieves high security as NIST SP 800-22 standard requires.
\end{itemize}
\end{resume}

\section{HONORS AND AWARDS}
\begin{itemize}
\item Fisher Scholarship, Computer Science Department, Oklahoma State University, 2014.
\item Honorable Mention, Mathematical Contest in Modeling (MCM) (top 25\%), 2011.
\item First Class Scholarships of Harbin Institute of Technology (top 5\%), 2008-2011.
\item Excellent Graduate of Harbin Institute of Technology (top 10\%), 2012.
\item Excellent Student of Harbin Institute of Technology (top 5\%), 2011.
\item First Prize, Chinese High School Physics Competition (top 5\%), 2007.
\end{itemize}

\end{document}